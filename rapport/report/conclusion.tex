Le projet d'informatique répartie nous permet de développer un projet de A à Z. Il nous force à faire des choix technologiques en fonction des spécifications et de la conception que nous avons réalisé auparavant. De cette façon, ce projet nous a appris l'importance des étapes de conception et de spécification. 

Les premières étapes nous ont forcé à faire des choix de spécifications et de conception, ce qui est différent de la plupart des projets auxquels nous avons participé jusqu'ici.

La mise en pratique des concepts abordés lors des cours d'informatique répartie nous a permis de mieux comprendre les technologies utilisées. 