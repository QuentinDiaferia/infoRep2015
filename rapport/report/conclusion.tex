Le projet d'informatique répartie nous permet de développer un projet de A à Z. Il nous force à faire des choix technologiques en fonction des spécifications et de la conception que nous avons réalisé auparavant. De cette façon, ce projet nous a appris l'importance des étapes de conception et de spécification. 

La plupart des difficultés techniques que nous avons rencontrée se trouve au niveau du développement en Java, notamment sur l'interface graphique. Nous avons dû élaborer les spécifications alors que nous n'avions que peu de connaissances sur les outils mis à notre disposition. Nous avons rencontré quelques problèmes organisationnels notamment à cause du nombre de membres de l'équipe. 

Cependant, la mise en pratique des concepts abordés lors des cours d'informatique répartie nous a permis de mieux comprendre les technologies utilisées.