Notre projet d'informatique répartie portait sur un bureau virtuel partagé entre plusieurs machines. Ce rapport présente les spécifications, la conception ainsi que les choix de réalisation de ce projet.

Tout d'abord, les spécifications présentent l'ensemble des éléments devant être présents dans la livraison finale. Elles présentent également les différents cas d'utilisation de l'application.

La conception, qui découle des spécifications, décrit les éléments principaux de l'application et leurs interactions. 

Après cette étape de conception, des choix techniques sont ressorti comme étant évidents pour la réalisation du projet. Les problèmes rencontrés et les solutions trouvées à ceux-ci sont également présentées dans ce rapport.

Dans ce rapport, les documents de spécifications et de conception ont été mis à jour. Les suppressions seront en rouge, les modifications en jaune et les ajouts en vert.

Pour lancer l'application en \textbf{localhost}, il faut d'abord utiliser le \textit{makefile} pour compiler les fichiers. Ensuite, il faut lancer le serveur avec le script \textit{execServeur} puis le(s) client(s) avec \textit{execClient}. 

Pour pouvoir faire fonctionner l'application en réseau, dans \textit{execClient}, il suffit de changer l'adresse IP pour qu'elle corresponde à celle de la machine serveur.