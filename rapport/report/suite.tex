\section{Difficultés rencontrées}

\subsection{Difficultés techniques}

\paragraph{}
L'une des difficultés que nous avons rencontrées, au niveau du développement, se situe au niveau de l'interface graphique. En effet, réaliser une interface graphique en Java, à l'aide de Swing, est une pratique que nous avons vue en cours l'année dernière, mais peu pratiquée : il a fallu nous replonger dans l'utilisation de cette librairie afin d'obtenir un rendu nous satisfaisant.

\paragraph{}
De plus, au niveau du Java également, la gestion de l'événementiel a été fastidieuse. Une fois encore, nous avons rapidement manipulé les différentes classes Java relatives aux événements il y a un an.

\paragraph{}
Au final, les problèmes majeurs de développement se sont situés au niveau de ces parties du programme et non de l'informatique répartie en elle-même, ce que nous regrettons.

\subsection{Difficultés organisationnelles}

\paragraph{}
L'un des problèmes majeurs auxquels nous avons dû faire face se situe au niveau de la communication au sein de l'équipe. En effet, le travail a du s'effectuer avec une équipe nombreuse, avec six membres, sans avoir à notre disposition une salle comme c'est le cas pour les PIC. Il nous a donc fallu utiliser des outils de gestion de projet et de communication afin de nous tenir au courant du travail des uns et des autres.

\paragraph{}
Nous avons donc utilisé les plateformes Trello, basée sur le principe du Scrumboard, et Slack, afin de nous envoyer des messages sur le projet en dehors des créneaux de cours dédié à celui-ci. Cependant, du fait de notre méconnaissance de ces outils, leur usage n'a pas été optimal.

\paragraph{}
L'absence de hiérarchie au sein de l'équipe a également posé quelques problèmes. En effet, en l'absence de chef, il n'y a eu aucun management de l'équipe, aucun leader permettant à l'équipe de se souder. Un tel rôle aurait permis d'éviter quelques dépassements de dates d'échéance que nous nous étions fixées, quelques soucis de communication, etc. 

\section{Améliorations possibles}

\paragraph{}
Notre projet, bien que fonctionnel, comporte plusieurs défauts et lacunes qui pourraient être corrigés dans le cadre d'une poursuite du projet :
\begin{itemize}
	\item La gestion des erreurs peut être grandement améliorée, via la création d'un ensemble d'exceptions et l'utilisation de fenêtres d'alertes ;
	\item Du fait du rafraîchissement du bureau tout entier lors de la modification d'un élément, le programme pourrait devenir assez lent en cas de rajout d'utilisateurs ou de widgets : il serait alors intéressant d'affiner la mise à jour du bureau en gérant indépendemment chaque widget ;
	\item Nous avons du abandonner le widget calculatrice par faute de temps.
\end{itemize}

\paragraph{}
De plus, plusieurs fonctionnalités intéressantes peuvent être ajoutées :
\begin{itemize}
	\item Il est aisément possible de rajouter des widgets, selon l'imagination des développeurs ;
	\item De fonctionnalités relatives à l'interface graphique peuvent être ajoutés : redimensionnement et réduction des fenêtres, gestion de raccourcis sur le bureau, ou toute autre fonction s'inspirant des bureaux d'OS comme Ubuntu ou Windows ;
	\item Nous pourrions également imaginer un système de discussion instantannée entre les clients leur permettant de discuter pendant l'utilisation du programme.
\end{itemize}


