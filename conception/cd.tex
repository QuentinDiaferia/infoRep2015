\color{red}
\section{Algorithme de la procédure : connect (supprimé)}
\begin{algorithme}
	\small
	\procedure{connect}{\paramEntreeSortie{nbUtilisateurs: \naturel}; \paramSortie{ConnexionReussie: \booleen, numeroUtilisateur: \naturel}}
	{}
	{
	  	\sialorssinon{nbUtilisateurs $<$ 4}
	  	{
	  		\affecter{nbUtilisateurs}{nbUtilisateurs+1}
	  		\affecter{ConnexionReussie}{Vrai}
	  		\affecter{numeroUtilisateur}{genererNumUtilisateur()}
	  		\instruction{sauvegarderUtilisateur(numeroUtilisateur)}
	  	}
	  	{
	  		\affecter{ConnexionReussie}{Faux}
	  		\instruction{throw new TooManyUsersException()}
	  	}
	}
\end{algorithme}

\section{Algorithme de la procédure : openWidget (supprimé)}
\begin{algorithme}
	\small
	\procedure{openWidget}{\paramEntreeSortie{nbWidgetActives: \naturel};  \paramEntree{numeroWidget: \naturel, numeroUtilisateur: \naturel}; \paramSortie{ouvertureReussie: \booleen, widget: Widget}}
	{}
	{
	  	\sialorssinon{nbWidgetActives $<$ 6}
	  	{
	  		\affecter{nbWidgetActives}{nbWidgetActives+1}
	  		\affecter{widget}{new Widget(numeroWidget,numeroUtilisateur)}
	  		\affecter{ouvertureReussie}{Vrai}
	  	}
	  	{
	  		\affecter{ouvertureReussie}{Faux}
	  		\instruction{throw new TooManyWindowsException()}
	  	}
	}
\end{algorithme}

\section{Algorithme de la procédure : compute (supprimé)}
\begin{algorithme}
	\small
	\fonction{compute}{operande1, operande2: \naturel, operateur: \caractere}{\naturel}
	{}
	{
	  	\sialorssinon{operateur = "-" et  operande2 $>$ operande1}
	  	{
	  		\instruction{throw new CalculException()}
	  	}
	  	{
	  		\sialorssinon{operateur = "-"}
	  		{
	  			\retourner{operande1-operande2}
	  		}
	  		{
	  			\retourner{operande1+operande2}
	  		}
	  	}
	}
\end{algorithme}

\section{Algorithme de la procédure : diaporama (supprimé)}
\begin{algorithme}
	\small
	\procedure{diaporama}{\paramEntreeSortie{galerie: Galerie}; \paramEntree{changement: \chaine}}
	{numImageActive, nbImages: \naturel}
	{
	  	\affecter{numImageActive}{galerie.getNumImage()}
	  	\affecter{nbImages}{galerie.getNbImages()}
	  	\sialorssinon{changement = "suivant"}
	  	{
	  		\sialorssinon{numImageActive $<$ nbImages}
	  		{
	  			\instruction{galerie.setImage(numImageActive+1)}
	  		}
	  		{
	  			\instruction{galerie.setImage(1)}
	  		}
	  	}
	  	{
	  		\sialorssinon{numImageActive = 1}
	  		{
	  			\instruction{galerie.setImage(nbImages)}
	  		}
	  		{
	  			\instruction{galerie.setImage(numImageActive-1)}
	  		}
	  	}
	}
\end{algorithme}

\section{Algorithme de la procédure : disconnect (supprimé)}
\begin{algorithme}
	\small
	\procedure{disconnect}{\paramEntreeSortie{nbUtilisateurs: \naturel};\paramEntree{numeroUtilisateur: \naturel}}
	{}
	{
	  	\sialors{demanderConfirmation()}
	  	{
	  		\affecter{nbUtilisateurs}{nbUtilisateurs-1}
	  		\instruction{supprimerUtilisateur(numeroUtilisateur)}
	  	}
	}
\end{algorithme}
\color{black}


\color{ForestGreen}
\section{Algorithme d'émission des données par le client}

\begin{algorithm}[H]
	\KwData{Bureau bureau, Client client}
	\KwResult{ - }
	\While{1}{
		\eIf{miseAJour(bureau)}{
			envoyerBureau(bureau, client)\;
		}{
			sleep(10)\;
		}
	}
\caption{Représentation pseudo-code du procédé d'émission des données par le client}
\end{algorithm}

\section{Algorithme de réception des données par le client}

\begin{algorithm}[H]
	\KwData{Bureau bureau, Bureau retour}
	\KwResult{ - }
	\While{1}{
		retour = recupererBureau()\;
		\If{existe(retour)}{
			bureau = retour\;
		}
	}
\caption{Représentation pseudo-code du procédé de réception des données par le client}
\end{algorithm}

\section{Algorithme d'émission des données par le serveur}

\begin{algorithm}[H]
	\KwData{Bureau bureau, Liste sockets}
	\KwResult{ - }
	\If{taille(sockets) \textgreater 0}{
		\While{existe(sockets.prochain)}{
			socket = defiler(sockets)\;
			envoyerBureau(bureau, socket)\;
		}
	}
\caption{Représentation pseudo-code du procédé d'émission des données par le serveur}
\end{algorithm}

\section{Algorithme de réception des données par le serveur}

\begin{algorithm}[H]
	\KwData{Bureau bureau, Bureau retour}
	\KwResult{ - }
	\While{1}{
		retour = recupererBureau()\;
		\If{existe(retour)}{
			bureau = retour\;
			envoyerBureau(bureau)\;
		}
	}
\caption{Représentation pseudo-code du procédé de réception des données par le serveur}
\end{algorithm}
\color{black}
