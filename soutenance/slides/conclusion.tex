% Grégoire
% Conclusion

%%%%
% Pourquoi tant d'améliorations possibles ? Et bien, 
% essentiellement parce que le déroulement du projet nous as 
% mis face à des problèmes difficiles. C'est tout ce qui a fait
% de ce projet un projet enrichissant. En effet, il nous a par 
% exemple, permis de comprendre l'importance capitale de la phase
% de conception dans un projet informatique. 
% Ce fut aussi, comme d'autres l'ont déjà dit, une mise en 
% pratique du cours d'informatique répartie. Dorénavant, les
% sockets ne me font plus peur !
% Enfin, et nous terminerons là-dessus, c'est un projet qui est
% intéressant de par sa vocation à fournir des problématiques
% réalistes et existantes en entreprise. 
% Merci pour l'attention que vous avez portée à notre travail ;
% nous pouvons désormais répondre à vos questions.

\begin{frame}
	\frametitle{Conclusion}
	\begin{itemize}
		\item Problèmes enrichissants 
		\item Mise en pratique du cours d'informatique répartie
		\item Problématiques réalistes et intéressantes
	\end{itemize}
\end{frame}