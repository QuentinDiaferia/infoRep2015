Le bureau propose quatre types de widgets : calculatrice, météo, galerie, bloc-notes.
{\color{red}
\subsection*{Calculatrice}

Ce widget est constitué d'un champ de saisie des calcul, d'un champ d'affichage du résultat d'un calcul et de boutons de saisie des chiffres ou opérateurs de calcul.

Il permet la réalisation de calculs simples : addition et soustraction d'entiers naturels.

Il permet également l'effacement de l'opération en cours de saisie à l'aide d'un bouton effacer qui efface la totalité de la ligne de saisie.

Le widget possède donc les éléments suivants :
\begin{itemize}
\item un bouton "+";
\item un bouton "-";
\item un bouton "effacer";
\item un pavé numérique.
\end{itemize}
}
\subsection*{Météo}
Ce widget indique la météo d'une ville en fournissant les informations suivantes : nom de la ville, température, tendance météorologique.

Une seule ville sera présentée par ce widget, Rouen, avec des données figées (sans mise à jour en temps réel) sur le serveur et non modifiables par les utilisateurs.

\subsection*{Galerie}
Ce widget est constitué d'une zone d'affichage des images et de boutons de navigation (gauche et droite) permettant le défilement des images de la galerie en boucle infinie.

Les images sont stockées sur le serveur et ne sont pas modifiables par les utilisateurs.

\subsection*{Bloc-notes}
Ce widget est constitué d'une zone de texte modifiable par une entrée standard au clavier et dans laquelle le positionnement peut être fait à l'aide d'une souris ou du pavé directionnel du clavier.