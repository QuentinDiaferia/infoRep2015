% L’utilisateur connecté au bureau virtuel pourra lancer un widget au travers d’un lanceur. Celui-ci s’apparentera au dock du système 
% d’exploitation Mac. Cela ne sera possible que si le nombre de widgets ouverts n’a pas atteint sa valeur maximale.

% L’utilisateur pourra ouvrir une fenêtre de type calculatrice et saisir une addition ou une soustraction. Le résultat apparaîtra sur 
% l’écran de tous les utilisateurs.

% Un utilisateur pourra aussi lancer le widget bloc-notes, permettant de saisir du texte et ainsi de l’afficher auprès des autres machines clientes.

% Un widget « Photos » pourra être utilisé afin de faire défiler quelques images pré-chargées sur le serveur.

% L’utilisateur connecté au bureau virtuel pourra quitter l’application via un bouton disponible dans le dock de lancement.

\subsection{Scénario d'utilisation 1}
\begin{itemize}
	\item Nom : 
	\item Description : 
	\item Acteurs : 
\end{itemize}

\paragraph{Séquence d'événements}
\begin{itemize}
	\item ;
	\item .
\end{itemize}

\paragraph{Exceptions}
\begin{itemize}
	\item ;
	\item .
\end{itemize}

\subsection{Scénario d'utilisation 2}
\paragraph{}

\subsection{Scénario d'utilisation 3}
\paragraph{}

\subsection{Scénario d'utilisation 4}
\paragraph{}

\subsection{Scénario d'utilisation 5}
\paragraph{}

