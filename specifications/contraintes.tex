Les performances du produit livré à l'issue du projet sont très 
importante. Le logiciel conçu devra, pour simuler une application
temps-réel, avoir un temps de réponse infiniment petit. Le temps
de réponse maximum est de 2 secondes (hors latence liée au réseau).

En outre, l'application en production doit supporter l'ouverture de 
six fenêtres au maximum et de quatre utilisateurs connectés simultanément.

Pour des raisons légales, il sera impossible d'importer ses propres
photos dans la fenêtre de galerie de photos. L'application n'accèdera
ainsi à aucun moment aux données personnelles stockées sur les machines
des utilisateurs.

Lors du développement, une attention toute particulière sera portée
à la gestion du système de verrous sur des fenêtres. En effet, pour 
simplifier l'utilisation de l'application, les utilisateurs ne 
pourront pas avoir la main sur plus d'une fenêtre à la fois.

Suite à la demande du client, il est demandé de privilégier, tant que
possible, l'utilisation de méthodes liées à l'informatique répartie pour
la gestion du réseau dans l'application.

